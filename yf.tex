\documentclass[UTF8]{ctexart}

\usepackage{geometry}
\geometry{a6paper,centering,scale=0.8}

\title{\heiti 医方}
\author{\kaishu Asd编辑}
\date{\today}

\begin{document}
\maketitle
\tableofcontents

\section{医方总论}
势有春夏秋冬,药有温热凉寒。温热之剂皆为补虚,凉寒之剂皆为泻实。

热者寒之,寒者热之,虚则补之,实则泻之,乃万病治法之总纲。临床必先明察病之寒热属虚实,而后乃可投药取效也。

热者寒之,其热火也。火有内外虚实之异。所谓外火者,外入之火也,属实火;其在表者,当辛凉以散之。薄荷、葛根、柴胡之属也。银翘饮桑菊饮皆主之。其在里者,苦寒辛寒咸寒以折之。苦寒者,芩连扼柏之属也。黄连解毒汤,葛根芩连汤皆主之。咸寒者,硝黄之属也。三承气主之。辛寒者,石膏之属也。白虎汤主之。内火者,内生之火也,乃阴血亏损,虚火妄动所致也。二地,二冬,元参、石触、龟板、鳖甲白芍之属也。肺胃阴虚主以沙参麦冬汤,益胃汤亦主之。肝肾阴亏者,重则大小定风珠,三甲复脉。轻则六味丸。无论外入之实,或内生之虚,大忌温补。误用者,尤抱薪救火也。

寒者热之,其寒亦有内外之分,外寒,乃外入之寒,属实。其在表者,辛温以散之,其入里者,当温热折之。辛散者,麻桂荆防香薷之属也。轻则荆防败毒散,重则麻黄汤桂枝汤。怯寒者,姜附之属也。四逆汤主之。内寒者,内生之寒也,属虚。宜辛温辛热以补之。轻则参芪草,重则姜附。虚寒之轻症,诸君子汤主之。重症,非四逆不可为功也。无论内生之虚或外入之实,皆当温补,大忌凉寒,误用无异于雪上加霜也。

实则攻之,乃正胜邪强,伐邪保正之法也。正气盛,攻之则邪退正安,误补则邪恋难去。清汗下三法,皆主之。邪在卫分,桑菊饮银翘散皆主之;气分受邪者,白虎汤,葛根芩连汤,诸承气汤,凉膈散皆主之。营分受邪,清营汤主之;血分受邪,则主以犀角地黄汤。下法又有寒下温下润下之不同。热结者,宜寒下,大小承气主之;寒结者,宜温下,三物备急丸主之,温脾汤亦主之。汗法,有辛温辛凉之不同,辛温者,轻则荆防败毒散,重则麻黄桂枝汤。辛凉则轻用银翘桑菊,重则麻杏石膏汤。

虚则补之,乃正弱邪强,保正御邪之法。劳倦伤阳,谋虑伤阴。湿燥寒之阴邪伤阳,亦间有宜补阴者;感风署火阳邪者伤阴,亦间有宜补阴者。先天之阳亏者,补命门,后天之阳亏者温胃气,先天之阴亏者补肾水,后天之阴亏者,补心肝。命门火衰者,八味丸主之。胃气伤者,四君子主之。肾水亏者,六味丸主之。心肝血虚者,四物汤主之。实则泻之,虚则补之,寒者热之,热者寒之,乃千古不易之治病大法。所言泻者,泻其有余是泻邪也,以邪气反盛正气不足,当却邪以卫正。所言补者,是补其不足,是补正也。因正气不足,邪从虚入,当扶正以去邪。四法乃万病之纲纪。医方虽盈千累万,皆依此而立。今余以法统方,虽不能尽括诸方,亦可从中窥其大端也。

\section{去风之剂}
风有自内而生者,有自外而入者。自内而生者谓之内风,多源自肝肾阴亏,属内伤不足之虚症。亦间有因外感阳热亢极而引发之实热。自外而入者,因风自外来,谓之外风,皆为外感有余之实症。六气风淫为首,故以风病尤多,其浅者只在皮毛则为伤风,其久者留于关节,为倭为痹;其深者入于脏腹,则为中风。肝风内动,则有眩晕震颤、四肢抽搐、足废不用、语言骞或卒然昏扑,人事不省,或口眼歪斜,半身不遂诸侯现焉。内风有虚风和实风之分。实风因为阳盛极,虚风则源于肝肾(阴)虚。外风者,风自外入,常夹它气伤人,中外风则头痛恶风,肌肤瘙痒、肢体麻木、筋骨栾痛、关节不利,甚者口眼歪斜,角弓反张诸侯生焉。内风外风,源自各异,制法各异,外风宜疏散,内风宜平息也。外风可引动内风,内风也可兼杂外风也。

先以外风论之,肉不坚,腠理疏者易得之。风者,阳邪也,善行而数变,常夹它气伤人,挟寒则谓之风寒,挟热则谓之风热,挟湿则谓之风湿。风寒、风热、风湿伤人肌肤、经络、筋骨,皆属表证,宜以辛温辛凉之药散之,而兼以散寒清热去湿可也。因三者皆为轻症,故统称伤风。风寒湿三气杂至,而肢体麻木,关节酸痛,谓之痹证也。风邪上涌头目,而治急性头痛,谓之头风;素体阳虚之人,卒为爆风所中,而为卒倒昏扑、口眼歪斜、半身不遂、外有六经形证,则谓之真中风也。伤风、头痛、真中风、痹症四种统称外风。伤风挟寒者。谓之风寒,荆房败毒散、麻黄汤主之。伤风而挟热者谓之风热,桑菊饮、银翘散、麻杏石膏汤皆主之。伤风兼湿者谓之风湿,羌活胜湿汤主之。头风者,三痹汤和川工调茶散、菊花调茶散皆主之。

再以内风论之,内风之起,或因于外感寒热之邪,阳热亢极,引动肝风而为实证。或因于肾水内亏、肝虚血少而致虚风内动,为虚证。虚风又有因失血过多,房劳过度得之者,有因热病久耗,伤阴烁血得之者,以羚羊钩藤汤、镇肝息风汤主治;因外感热邪伤阴者,主以阿胶鸡子黄汤,大小定风珠主之。

\section{去寒剂}
寒为阴邪,耗气伤阳,得温热可解,以热可胜寒也。

寒有内外之分,外寒寒在肌表,内寒寒在脏器。外寒,寒自外入,属有余之外感,皆为实证;内寒则虚实并见。实者,有余证也,寒邪自外而入直中三阴(太阴、少阴、厥阴)属之,谓之实寒;虚者,不足证也,阴盛阳亏者属此,谓之虚寒。实寒证,或客于肌肤之表或直客三阴之里,皆为外感,独阴盛阳虚之内亏证,为内伤。

先以内寒而论之,寒邪以风为载体,外束肌表,遂有恶寒、发热头痛、喘满之症,当以辛温解表,使寒随汗泄,麻黄汤,荆防败毒散、人参败毒散皆主之。

内寒之治,不论虚实,阴盛或阳衰,概以温热折之。内寒,寒在三阴者病轻,寒在周身者病重。寒轻者,治在助阳,寒重者,治在回阳。太阴虚寒,则胸满呕恶,腹部胀痛,纳差便溏,肢冷脉迟,理中汤大小建中皆主之。厥阴虚寒,则有腹痛、呕恶之侯,吴茱萸汤主之。少阴虚寒,则半身以下切冷、腰软膝弱,尺买迟小之侯,肾气丸主之。脾肾皆寒,五更泄泻者,主以四神丸。周身寒重,阳气衰微欲脱者,厥逆汗出,气促痰喘者治以温热回阳。回阳,主以四逆,独参汤或参附龙牡救急汤,参附汤皆主之。去寒剂,非温即热,为阳盛之实热和阴亏之虚热证之大忌,误用无异于抱薪救火也。

\section{去暑剂}
暑为阳邪,乃夏令之主气,暑气通心,而最易耗气伤阴。先夏至而病者谓之病温,后下至而病者谓之病暑。 

暑为夏令之主气,乃温病之一。然暑必夹湿,故暑病实乃温病中之湿热病也。清利湿热乃暑病攻邪之法。而益气养阴则为暑病扶正之法也。其病有热重湿重之别;虚证则有气耗阴伤之异。暑病之实证,热重湿轻者,湿易从火化,去湿不可过于温燥,以免有伤阴助阳之痹也。湿重热轻者,则暑为湿涸,用药不可过于凉寒。暑病之治,有清热、利湿、益气、养阴之四法。清热者,黄连石膏之暑也;利湿者,滑石、竹叶、薏米、茯苓之属也;益气者,人参甘草之法也;养阴者,人参麦冬之属也。

骄阳暴晒或高温做业,以致暑热客于肌体,高温不得放散,而显汗多、烦渴、头昏、头痛、发热、呕恶诸证,谓之阳暑证。阳暑证之重者,谓之中暑,当以竹叶石膏汤,白虎人参汤、苍竹白虎汤主治;阳暑证之轻者,谓之伤暑,以益气养阴汤主治。

若夏令炎热,贪凉过度。或过食生冷,内伤水湿而显头痛、无汗、形寒、身热、或兼见腹痛、腹泻诸侯者,则谓之阴暑证。阴暑证实乃暑气内伏,外为寒束之症也,当以三味香薷饮、黄连香薷饮、新加香薷饮、六味香薷饮、十味香薷饮主治。

另暑气通心,故病暑者,每见卒倒昏扑人事不醒之见证,当以卫生防疫保丹主治。

暑为热病,易耗气伤阴,热者脉盛,气阴伤者脉细,夹湿重者脉薷,故暑病之脉或薷或细而兼数也。

\section{去湿剂}
湿为阴邪,其性重滞,其中人缓。湿之与水异名同类,湿为水之渐,水为湿之积。人身之中主水在肾,制水在脾、调水在肺。故水湿病与肺脾肾息息相关也。脾虚则湿生,肾虚则水泛,肺失宣降,则水津失布也。

湿有内湿外湿之分。内湿者,内生之湿也,源自脾肾亏损;外湿者,外入之湿也,或因居处卑湿、天雨湿蒸或源自冒雨涉水、汗出湿衣。盖内湿,湿自内生,乃内伤不足之症,病在脾肾;外湿,乃湿自外入于肌表经络,病在肺,乃外感有余之实证。内湿外湿源各不同,治各有异。外湿宜发汗,古则谓之开鬼门,麻黄、桂枝、羌活、独活、防风、蔓荆子之属也,代表方为麻黄汤、羌活胜湿汤。内湿宜利小便,古则谓之洁净腑,防己、赤小豆、猪苓、茯苓、泽泻之属也。代表方为五皮饮、五苓散、猪苓汤。

内湿之治除汗利外,苦燥温燥之法也为治湿之常法。湿从寒化而为寒湿者,宜温燥,平胃散、藿香正气散皆主之。湿从热化而为湿热者,则宜凉燥,二妙散主之。

外湿为外感寒湿之邪,客于肌表经络,而致发热恶寒,肌肉或筋骨疼痛之症。

内湿为脾肾内伤,脾虚不能运化水湿,肾虚不能化气行水,遂至水道不利,水液内停而成水湿之症也。

然肌肤与脏腑相表里,表湿可内传脏腑,里湿可外溢肌肤,外湿内湿可相间并见也。

盖外湿伤人,或挟寒或挟热。邪寒则谓之寒湿,挟热则谓之湿热。寒湿伤人,以风为载体,初客肌表,而为急性者,可以羌活胜湿汤、羌活除湿汤或神术散发之。舟车丸,疏凿饮子亦主之。寒湿久郁,肌表经络、肌肉筋骨因之而痛则成寒湿痹证之慢性者,当以三痹汤、独活寄生汤主之。外感湿热病于上焦者,以三仁汤、甘露消毒汤主之;病于中焦者主以连朴饮或霍朴夏苓汤;病于下焦者,主以八正散、此外,二妙散、茵陈蒿汤也主之。内湿水肿而脾虚失运者,首推实脾饮,次则真武汤、苓桂术甘汤;内湿而属气不化水者,则主以金贵肾气丸、附子理中汤二方主治。此外,五苓散、四苓散、猪苓汤、五皮饮亦为临床所常见也。

\section{润燥剂}
燥为阴邪,乃秋令之主气。最易伤阴,有内燥外燥之分际。外燥邪自外入,为外感之有余,实证是也。内燥,燥自内生,为内伤之不足,虚证也。二者虽同为燥,源不同,治各异也。外燥宜清宣,内燥宜滋润也。轻宣者,杏仁、桑叶、薄荷、桔梗之属也;滋润者,生地、元参、麦冬、玉竹、石除、天冬、天花粉之属也。

先以外燥而论之。燥乃秋令之主气。秋风初凉,西风萧杀,感之者多病风燥,属凉燥也。其侯恶寒、身热、咳蔌、头痛、鼻塞、唇燥、咽干,治宜温散肺寒,清凉润燥,杏苏散主之;若久晴无雨,秋阳以爆,感之者多属温燥,其侯身热、口渴、咽痛。咳蔌少痰、痰中带血,治宜清肺润燥,清燥救肺汤主之,沙参麦冬汤亦主之。温燥凉燥同外燥,但温燥为热而凉燥为寒也。

内燥或因于房劳,或因于热病劫阴(热病后期),或因于温药久服克伐太过。以病位言之,有上燥、中燥下燥之分际。上燥则上逆做咳,中燥则呕恶不食,下燥则消渴和大便秘结。上燥,病在肺,百合固金汤、麦门冬汤、养阴清肺汤皆主之。下燥燥在肠肾,肠燥主以增液汤、通幽汤、润燥丸;肾燥主以大补阴丸六味丸、益肾汤。

\section{去火剂}
火为阳邪,可耗阴伤血,而有内外之分际。外火,火自外入,为外感有余之实证,皆属实火。初客为实火,迁延转虚火。内火,火自内生,皆为内伤不足之虚证,病在脾肾,属虚火。治实火宜清宜泻,当以苦寒折之。虚火成因不同,治法各异:有因阴虚阳亢者,当滋阴以降之;有伤于饮食劳倦,虚寒中生者,当补土藏阳;有因命门火衰,虚阳浮越者,当引火归原。

先以内火而论,房劳太过,伤及真阴;或失血太过,虚火妄动,当滋其阴则火自降矣,以钱己六味丸为祖。劳倦伤阳,虚寒中生,寒极则土不藏阳,虚火浮越,法当补土藏之,补中益气汤主之。肾阳虚衰,命火不足,则阳虚阴盛,而为肾经虚寒,寒极则虚阳浮越,而成真寒假热证,法当引火归原,以消其阴翳也,以肾气丸为祖。

外火之来,有伤寒温病两大法门。

盖在表发散,在中和解,在里清下,乃伤寒三阳热病治法之总括。伤寒之邪内传阳明则化火矣。邪仅在阳明之经者,清之可也,葛根芹连汤、白虎汤皆主之。一入阳明之腑,则当下,三承气主之。

温病,则有卫气营血之传变。病邪仅在肺卫,则发热恶寒,当辛凉以散之,表邪一解,则诸侯自退矣。桑菊饮、银翘散俱主之。邪入气分,则有尿黄、便结、热渴、汗出之侯,清下可也。热郁胸膈者,凉膈散主之;热在肠胃,白虎汤、葛根芹连汤、三承气、白头翁汤皆主之。邪入营分,则有发斑、神昏、谵语之侯见焉,法当透热转气,以清营汤主治;邪在血分,则耗血动血,当凉血散血,犀角地黄汤主之。

终以脏腑而论之。心经火旺,心烦、口渴、口糜溲赤,用导赤散主治;火在肝胆,肋痛、口苦、便结、溲赤、湿热发黄者,龙胆泻肝汤主之;胃中有火,唇干、口臭,泻黄散主之;肺火咳喘,泻白散主之;肝肾阴亏,骨蒸劳热者,清骨散主之。

\section{发表剂}
外邪客表乃外感有余之症。治之当逐之使出,邪散则表解矣。解表者,用辛散之药开腠取汗,以速馈客表之外邪也。盖客表之邪不越寒热之两者,然必兼风为患。寒为阴邪,伤人阳气,挟风则谓之风寒,因之成病者谓之伤寒。热为阳邪,喜耗伤阴血,挟风则谓之风热,因之而成之病则谓之温病。伤寒解表宜乎辛温,麻桂荆防紫苏之属也,以辛可发散,温可胜寒也;温病解表宜乎辛凉,桑叶菊花、薄荷、连翘之属也,以辛可发散,凉可折热也。

发表剂,以辛温辛凉之药,开腠取汗,以解除伤寒、温病之表证也。夫外邪客表,则发热、恶寒、头痛、项强、脉浮,或有汗或无汗诸侯见焉。诸侯皆因邪客肌表所生,故统谓之表证,用以解除表证之剂则谓之解表剂也。

先以伤寒而论之,太阳主表,统一身之营卫,卫行脉外,风能中之;营行脉中,寒能伤之。风伤于卫则谓之中风;风先伤卫,复又伤营则谓之伤寒。中风属表虚证,其侯发热、恶风、头痛、项强、脉浮而有汗;伤寒为表实证,其侯发热、恶寒、腰痛、骨节烦痛,脉浮而无汗。中风则以桂枝汤调和营卫,汗止表解;伤寒则用麻黄汤发表,汗出则表解。伤寒无汗喘满而烦燥者主以大青龙。

再以温病而论之,热为阳邪,挟风则谓之风热,风热之邪自口鼻吸受肺卫,遂见发热重恶寒轻,气逆咳喘咽喉不利诸侯。轻则桑菊饮重则麻杏石膏汤。

总之解表有新温辛凉之二法,辛温宜乎风寒,以麻桂二汤为祖,后世之荆防败毒散,人参败毒散、华盖散皆自此出。辛凉宜乎风热,以桑菊饮银翘散为代表。麻杏石膏汤、升阳散火汤皆为此类。

本元素亏,为外邪所客,专事发汗则伤正,正伤而邪益深入矣。值此本虚标实之际,解表框正并施方为万全,若此则攻不伤正,补不留邪矣。盖人之正气,不越气血阴阳之四端。补气者,参芪之属也;补阳者,桂附之属也;补阴者,生地、元参、麦冬、熟地属;补血者,归勺之属也。然气阳同类,合而谓之阳气;阴血同类合而谓之阴血。素体阳亏,平素脉沉无力,面色苍白、身冷恶寒、表卫失调,易为寒邪所客,而成阳虚表证。若专以辛温发散,恐有亡阳之危,可以桂附助阳,以参芪益气,麻黄附子细辛汤,再造散悉为此类。素体阴亏,口干、心烦、渴而多饮、舌感苔少,甚则光剥,脉细数,易为温邪所客,而为阴亏表证。倘专以发散恐亡其阴,当解表去邪而参以元参熟地育阴,辅以归勺养血,加减葳仁汤则为此类。

\section{泻下剂}
泻下剂乃攻邪之剂。攻逐肠内因寒因热因液亏而积聚之宿食燥屎,使之自下窍泻出之剂也。

以其性而分有寒下,温下、润下之分,以其做用之强弱则有骏下缓下之别也。

所谓寒下者,攻热结之法也。邪热雍滞肠胃,结成燥屎,以致腑气不通。当以硝黄之寒凉,佐以枳实厚朴、青皮理气之药通气腑气,下之可也。仲景三承气,鞠通增液承气皆主之。

所谓温下者,攻逐寒结之法也。或直接以巴豆下之,代表方为三物备急丸。或以硝黄之寒凉,配以桂附之辛热合而下之, 温脾汤则为此类也。

所谓润下者,攻燥结之法也。宜于肠中津亏,大变燥结之症。盖肠道不能自亏,必有因以迫之,总不出阴虚血少之二端。当用麻仁、柏子仁、郁李仁润肠之品以治其标,兼用滋阴养血之物以图其本,麻子仁丸为其代表。

正亏邪实者,治当攻补兼施,攻之以去实邪,补之以扶正气。倘独用攻, 益伤正气;若不用下,必致拥塞而死。值此危亡关头,当攻补并施,乃可取邪去正存之效也。所以邪实正伤者,或因于正胜邪实之应下失下,或源于本虚而为邪气外乘。前者正伤于邪实之后,后者正伤于已实之前。当以黄龙汤主治。

下剂,为攻邪之法。唯病实者宜之。攻者,去其实也。但诸病之实有微甚,用攻之法分轻重。大实者,攻之未及可以再加;微实者,攻 之太过,每以治害,当慎也。凡伤寒者,病在阳者,不可攻阴;病在胸者,不可攻脏。若此,病必乘虚内陷,所谓引贼入门也。病在阴者,勿攻其阳,病在内者,勿攻其外,若此必因误而病加。所谓自撤番蔽者此也。盖实不嫌攻,但略加甘滞,便相牵制;虚不嫌补,若略加消耗,便觉相妨。实而误补,不过增病,病增可解;虚而误攻必先脱元,元脱无治矣。是皆攻法之大要也。

\section{和解剂}

凡以调和之法,解半表半里之邪、肝脾功能失调、上下寒热互结之症方剂,统谓和解剂也。属八法中和法。盖肝之与胆、脾之与胃皆互为表里之脏,受病每相互累及。脾与胃同属中州,肝与胆亦居中州,互相累及后遂成肝胃不和和肝脾不和之症也。

夫邪困少阳,遂见往来寒热,胸肋苦满、心烦善呕、默默不欲食以及口苦、咽干、目眩之侯。因邪居半表半里,即要透表,又要清里,又要防邪内入而加病,故以小柴胡合之。蒿琴清胆汤亦为斯类。

调和肝脾之药,宜于肝气郁结,横范脾胃;或脾之不运,致肝肝失疏泄。遂见而致胸闷肋痛、皖腹胀痛不思饮食、大便泄泻,甚则往来寒热等肝脾不和证候。以疏肝药,柴胡枳壳陈皮香附之属,与归勺之活血之品,与健脾助运药白术、甘草、茯苓之属组成,代表方四逆散、逍遥散、痛泻要方。

调和肠胃药,适于邪犯肠胃。寒热夹杂、升降失常而致心下痞、恶心呕吐、脘腹胀满、肠鸣下利诸证。以甘姜、黄芹、半夏辛开苦降为主,配入人参甘草补气和中药成方。代表方为半夏泻心汤。

\section{表里双解剂}
表里双解剂,以解表药,配以泻下、清热、温里药组成,有表里共图之作用,可治表里同病之方剂也。

邪在表汗之可也,使自表入之邪仍自表出。表寒者,辛温以散之,麻黄、桂枝、紫苏、白芷、荆芥、防风之属也;表热者辛凉以散之,柴胡、葛根、豆鼓、薄荷之属也。邪在里者清之下之温之可也。里热无结宜清,三黄、石膏、大青叶、鱼腥草之属也。里热有结者,宜下硝黄之属也。里寒则但宜温补,肉桂干姜之属也。盖表未除里又急者,徒散表则里邪不解,专攻里则在表之邪不去。万全法,解表攻里并用,乃可取表解里安之速效。临证当审表里而全衡用药,方 无太过不及之虞也。

临床分解表攻里、解表温里、解表清里三法存焉。

解表攻里,一于外有表证里有热结之症,大柴胡汤,防风通圣散皆为此类。

解表清里,宜于表证未解,里证又炽者,三黄石膏汤、葛根芹连汤为其代表方。

解表温里,宜于表里俱寒,表证未解里证又现之症。五积散为其代表。

\section{去痰剂}
痰者,津液所化也,乃津液之病名。以病因论,不出外感、内伤之两端;以病位言,或病于肺,或病于脾肾。

先以外感论,外感之痰,得之风寒暑湿燥。盖外邪来客,肺卫首当其冲,肺气受困,失于宣降,津液因之失布,聚而痰浊成矣。治之,当疏风宣肺,肺气通则痰自消矣。盖外邪客肺,风为载体,故统称风痰。挟寒谓之风寒,挟热谓之风热,挟湿谓之风湿,挟燥则谓之风燥也。因于风寒者,治当疏风散寒,三拗汤、麻黄汤、荆防败毒散、止咳散、杏苏散、人参败毒散皆主之。因于风热者,治以疏风散热,银翘散、桑菊饮、麻杏石膏汤皆主之。因于风湿者,当去风胜湿,羌活胜湿汤主之。因于风燥者,治当疏风润燥,清燥救肺汤,沙参麦冬汤皆主之。火郁于肺,炼液为痰,则谓之热痰,治当清化,清气化痰丸,清金化痰丸皆主之。此皆属外感客肺。然肺经内伤者亦不鲜见,内伤者何?肺燥阴伤,虚火烁金成痰,沙参麦冬汤主之。总之肺经受病,外感内伤兼有,病于外感者,所感为六淫,病于内伤者,所伤在津液。

脾肾生痰,纯为内伤。

脾气弱,则津液失布,聚而痰成,谓之湿痰。其病脉缓,面黄肢重,腹胀食滞,痰滑易出。治当温脾胃固中气,治本以桂苓术甘汤为祖,治标则以二陈汤为祖。后世温胆汤、涤痰汤、导痰汤皆源于二陈。

肾有水火二脏,两者病,皆可生痰、

水者,肾阴也。肾水一亏,则阳强火动,虚火炼津为痰则痰浊生焉,所谓水沸为痰者此也。  火者肾阳也,真阳虚弱,命火衰微,不能摄服其水,则如洪水不能归源,逆流泛滥而为痰,所谓水泛为痰者此也。当以肾气丸益火之源。八味丸亦主之,加牛漆五味子更效。

痰之为物,随气升降,无处不在,或在脏腑或在经络,所以为病之多也。凡有怪病莫不由兹,故丹溪有十病九痰之论也。但治其所因,使津液各归其经,则非痰矣。盖痰随气升,气雍则痰聚,气顺则痰消,故善治痰者,不治痰而治气,气顺则一身之津液随气而顺矣。故此去痰之剂,多配理气之品也。

痰之本,水也,源于肾;痰之动,湿也,主于脾。脾虚湿动则生湿痰,肾虚水泛则为痰饮。实寒、虚寒、实热、虚热皆可生痰。寒者温之,湿者燥之,风则散之,燥则润之、硬则软之,则治痰之法毕也。痰为病,必有因以致之,因风因火而生痰,但治其风火,风火息而痰自清矣。因虚因实而致痰者,但治其虚实,虚实愈而痰自平矣。未有治其痰而风火自撤,虚实自调者。以痰必因病而生,非病之因痰而致也。治之当详审虚实,有余之实痰可行消伐,不足之虚痰但宜调补,若妄攻无不危矣。总之,治痰之法,但能使元气日强,则痰必日少。若元气日衰,则水谷津液无非痰也。随去随生,有能攻之使尽,而且保元气无恙乎?故善治痰者,唯能使之不生,方是补天之手。倘不辨虚实概以攻之,痰暂去而复生,正气益困,痰气愈多,使病者稍宽于一时,实遗患于日后。

\section{理血剂}
理血者,活血止血也。淤血当活,血溢当止。

先就淤血证论之。

血脉环流,内溉五脏六腑,外养四肢百骸。寒热外邪内侵或本元虚亏皆可生寒成热。淤血之生不出此寒凝热结之两端。然血附气而行,气行则血行也。故活血药中必配行气之品。活血行气血淤证之大法。活血者,当归川弓、桃仁、红花赤芍、丹参之属也。四物汤乃活血之祖,后世类方(桃红四物、膈下逐於、通窍活血汤、补阳活血汤)皆自之出。行气者,郁金柴胡、陈皮、枳实香附之属也。以四逆汤为祖,后世行气类方皆出于此。临证当详审病源之虚实寒热,主以行气活血以除其标,佐以温经散寒、凉血清热、益气养血以图其本。外寒内之寒凝当温经益气以图其本,右归丸主之。因外热内客营血而致热结者,当凉血清热以图其本,犀角地黄汤主之。因阴亏内热而致热结者,当滋阴养血以图其本,六味丸主之。

止血,宜于血溢证,无论自身上溢之咳血、吐血、吜血、或血从下溢之便血、尿血、血崩统归出血。血溢当止,止血者,侧柏叶、灶黄土、艾叶之属也。血之外溢或源自热破血行,或生自脾虚不统。热迫血行,当分虚实,起病急,病程短、色红而急;虚者,起病缓,病程长,色暗而出血棉绵。

其治疗大法,不越以凉血为主,活血止血为辅。临证当详审病之在在肝在心在肠,而辅以清凉。尿血以小计饮子主治,便血以葵花散、秦韭白术丸主治;咳血,以石灰散主治,脾不统血,主以归脾汤。出血在上大忌升提升柴禁用,可灼用牛漆代储石引血下行;出血在下,慎用沉降,可配升柴之升提引血上行。

\section{理气剂}
凡以理气药组成,有升旗降气之做用,以治气滞、气逆之方剂,统称理气剂。

气为一身之主,升降出入周身,外以养四肢百骸,内温五脏六腑。倘劳倦过度或情志失调或饮食失节或寒温不适,皆可使气之升降失调,遂法气滞气逆之症。气滞者,治在行气解郁,气逆者治在降气平冲。二者每相兼为病,故而者多相间并用也。然病有实虚,行气降气之品多耗气伤阴。故多配以补气药以补其耗。病有主次方有专攻,所以据功用之异而分之为行气降气之两类。

年老体弱或有血证者,恐有耗气伤津之虞也。

先以气滞而论。

先以气滞而言。

上焦风寒外感,肺气失宣;中焦之痰食积滞,脾失健运,下焦之七情郁结,肝气失疏。皆可致之。无论其滞在肺在肝在脾,其或胀或痛之病则一也。气滞见于上焦则胸闷胸痛;见于中焦,则脘腹胀满,嗳气吐酸,呕恶食少,大便失常;见于下焦则肋胀腰痛少腹痛,女则月经不调或痛经。气滞者当行之使之通,行气者,陈皮、厚朴、木香、枳实、乌药、香附之属也,代表方为越鞠丸、金铃子散,半夏厚朴汤、枳实韭白桂枝汤,天台乌药散、暖肝煎、厚朴温中汤。

气逆者降之,肺气逆则咳,胃气逆则呕。肺气逆者主以苏子降气汤定喘汤。胃气逆者,主以旋复代诸汤、橘皮竹茹汤、丁香柿蒂汤主之。

\section{补益剂}
人身不过气血阴阳之四端,气阳同类,合谓阳气;血阴同类,合称阴血。血气对立,血为有形之物,气为无形之用。夫血者,气之配也,人之一身,五脏六腑四肢百骸,靡不借其营养也。血附气而行,随气畅逆。阴者,阳之配也;阳者,阴之附也。五脏之阴血,无不赖肾水以濡养;五脏之阳气,无不靠命火以温煦。阳阴互根,善补阳者,必于阴中求之,阳得阴助则生化无穷;善补阴者,必于阳中求之,阴得阳助,则源源不绝也。

心肝脾肺肾,乃五脏也。肾居其一。有水火二脏,水者,真阴也;火者,真阳也。肾中真火次第而上生啤土,脾土又上生肺金;肾中真水,次第而上生肝木,肝木又上生心火。此五脏相生之道也。故肾之为脏,合水火二气,为五脏六腑之根,乃先天之本;脾亦为脏,属土,位居中州,为气血生化之源,乃先天之本也。五脏亏损,于法当补。有补母和固本二法存焉。肝虚补肾,滋肾水可涵肝木也;脾虚补肾,益真火以生脾土也;肺虚补脾,补脾土以生肺金也。此皆虚则补母之法。所谓固本者,固护根本也。根本者何而为先天之本也。倘先后天因补益而获充盛,则诸虚百损皆因根本得固而告愈矣。

人身不过气血阴阳之四端,气虚者补肺脾,血虚者补心肝,阴虚者补肾水,阳虚者补命门。故补气血阴阳实则补五脏也。五脏亏损于法当补,然以补脾肾尤要。盖脾生后天气血,肾主先天阴阳也。

肺主后天之清气,脾主后天之谷气。补气者,补肺脾也。肺脾乃母子之脏,脾为土,肺属金,土生金,故脾土乃肺金之母也。肺子亏损,补其母子自壮矣。脾肺气虚,于法当补,补气者,参芪术草之属也。肺气虚者,主以玉屏风。脾气虚者,补以四君子、补中益气汤。气虚既可见少气懒言,语声低微、动则气促之肺经证候,亦可见大便溏泻,四肢困倦之脾经证候,倘气虚下陷,又可见脱肛、子宫下垂、小便失禁之侯矣。其脉或细软或虚大。当以补气为主,佐以升举,补中益气汤为治中气下陷之典型方剂。

心主血,而肝藏血。补血者,补心肝也。血虚于法当补。补血者,当归熟地、阿胶、首乌之属也。四物乃补血之祖,后世类方皆自之出。当归补血汤,归脾汤亦主之。阴气欲脱之大失血,而现肢厥自汗脉孔者,可以独参汤补之。血虚则面黄、口唇指尖仓白、头晕眼花、气促心跳诸侯生焉。

肾有水火二脏,主先天之阴阳。肾阴者,肾水也;肾中水涸,即不能上济以养肝木,而致肾阴亏损,又可上盗母气耗损肺阴。法当壮水之主,以制阳光。补阴者,地黄、龟板、枸杞、元参之属也。六味左归为代表,此外百合固金,秦艽扶羸汤皆为斯类。肾水亏则口干咽燥、便结溲黄、腰腿酸软、多梦不昧诸侯见焉。水不涵木,则头晕目眩。子盗母气,则咳痰音哑,骨蒸潮热,盗汗咯血。肾阳者,命火也。命门火衰,则不能上蒸脾土,脾阳亦弱;命火不足则畏寒惧冷、小便频数,阳道不举,脉象细软或沉迟、纳少呕恶、肠鸣便溏。法当益火之源以硝阴翳。壮阳者,肉桂、附鹿茸、紫河车、狗脊、巴戟天之属也。八味左归为其代表,附子理中丸,四神丸亦为此类。

气血两亏者,心肝脾肺共补,八珍汤主之。

阴阳两亏者,肾之水火同补,肾气丸主之。

气血阴阳皆亏者,五脏同补,十全大补汤主之。

\section{固涩剂}
人之四肢百骸,五脏六腑,无不赖气血津精以滋养也。其盈虚消长伴及终生,其病变有三:一曰太通,二曰不通,三曰亏损。亏损当补,不通宜通,太通宜涩也。

肺脾肾三脏气衰,窍髓松弛皆可治气血津精基础物质之滑脱。收敛固涩与补益精气之物相合,疗脏腑之衰退,复窍髓之松弛,以终固气血津精滑脱之方剂,统称为固涩剂。固涩者,涩可固脱也。方中以固涩药,治窍髓之松弛;以补益药充气血津精之损亏。临症当视病之缓急以调二药之比例,急则先治其标,缓则专图其本。至于元气爆脱之症,则当急以大剂补气回阳之物,急固其脱,方可挽垂危于顷刻也。盖五脏之经隧皆为肝主之筋膜组成,无论症主何脏,无不关乎肝经。故此病多肝肺同病,肝肾同 病,肝脾同病也。

气血津精之失因病脏之或肺或肾或脾之不同,而表现各异,或自汗盗汗,肺虚久咳,或遗精滑泻,小便失禁,或久泻久利,崩漏带下。久咳肺虚,宜补肺宣肺敛肺并行,补中寓宣,宣中兼补,始合肺司开合之机。表虚自汗宜益气实卫与敛汗潜阳同施,方合肺司卫分开合,肝司营分开合之机;中焦虚寒,肠滑失禁,宜温中健脾与涩肠止泻同用;肾虚失约小便不禁,宜温阳化气与补肾缩便兼顾,精关不固遗精滑泻,宜补肾与涩精同用;冲任不固带下崩漏,以补肾固冲与固涩止带收敛止血并举。固涩一法,本为久病之虚证专设,倘妄用于热病汗多,湿热下利,火动精遗湿热带下,或湿热所致之溲频数之实证,则有闭门留寇之患也。虚可固,实则不可;久则可固暴者不可。

本类方剂分固表止汗,敛肺止咳,涩肠致谢,涩精止遗,和固崩止带五类。

固表止汗宜于卫气不固之自汗症,或阴虚有汗之盗汗症,以益气固表之黄芪与敛汗之牡蛎等组成,玉屏风、牡蛎散、当归六黄汤皆为此类。

敛肺止咳,宜于久咳肺虚,气阴耗伤,以致喘促自汗,脉虚数之症。常用敛肺止咳药如五味子、乌梅等和益气养阴药人参阿胶等组成方剂。代表方九仙散、五味子汤。

涩肠固脱,宜于脾胃虚寒之久利。常用涩肠止泻药赤石脂、肉豆蔻、柯子、五味子与补肾药如补骨脂、肉桂、干姜、人参、白术等组成方剂。代表方真人养脏汤、四神丸、桃花汤。

涩精止遗宜于肾虚失藏,精关不固,遗精滑泻,或肾虚不摄膀胱失约之遗尿尿频。症属肾虚精遗者常用补肾涩精药茨实莲须等组成方剂,如金锁固精丸、;肾虚遗尿者常以固肾止遗药桑螵蛸益智仁等组成代表方剂桑螵蛸散、缩泉丸。

固崩止带用于妇人血崩暴注或带下淋漓之症,常用固崩止带药春根皮、黑荆芥赤石脂为主组成方剂,带表方固精丸。

\section{消导剂}
肖者散其积也,导者行其气也。脾虚不运则气不流行,气不流行则停滞为积。或做泻痢,或成微痞,以致饮食减少,五脏无以禀赋,气血日以虚衰,而致危困者多矣。

凡以消导药组成,具有消食导滞化积清微之功用,以疗时积痞块微瘕积聚之方剂,统称消导剂。属八法中消法范畴。凡气血痰湿食滞而成之积聚痞块皆可用之。消者,去其壅也,脏腑经络肌肉之间本无此物,忽而有之,必为消散乃得其平。

消导与攻下去有形实邪之用全同,而其缓峻则大异焉。泻下剂宜于病势急病程短者;消导剂则宜病势缓病程长者。倘急病,非攻不下,错投消导,病重药轻,其病难廖;若妄施攻下于渐成之积聚痞块,则骤伤其气,病反深固。盖积聚痞块薇瘕本为脾虚气滞所致固本逐邪方为正法,攻补并用方可取攻不伤正补不留邪之效。

脾虚气滞则积聚不化,故消导剂中常配理气药,气利以助积消。然后视病之寒热而投以对症之药。明辨病之虚实寒热缓急轻重,随症用药,方能合乎病情,取药到病除之功。

消食导滞剂为食积症的对之药。食积之为病,其候胸肋痞满,嗳腐吞酸,恶食呕逆,腹痛泄泻,常用消食药山楂神曲莱菔子为主组成方剂。脾胃亏甚者,需配以益气温脾之物,消补并施。代表方为健脾丸、枳术丸。

消痞化积剂则宜于微积痞块,此症多由寒热痰食与气血相博,聚而不散,日久为积。其候两肋微积,脘腹微结,攻撑做痛,饮食少思,肌肉消瘦,当以兴起活血化湿消痰软坚药组成,代表方为枳实消痞丸。

\end{document}
